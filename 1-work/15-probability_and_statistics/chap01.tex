%%%%%%%%%%%%%%%%%%%%%%%%%%%%%%%%%%%%%%%%%%%%%%%%%%%%%%%%%%%%%%%%%%%%%%%%%%%%%%%%%%%%%%%%%%%%%%%%%%%%
%	chap01 (確率)
%===================================================================================================
%	AUTHOR	H. NAKAJIMA
%	DATE	2025/01
%===================================================================================================
%	EDIT HISTORY
%	DATE		NAME			OVERVIEW
%---------------------------------------------------------------------------------------------------
%	2025/01		H. NAKAJIMA		Create new
%%%%%%%%%%%%%%%%%%%%%%%%%%%%%%%%%%%%%%%%%%%%%%%%%%%%%%%%%%%%%%%%%%%%%%%%%%%%%%%%%%%%%%%%%%%%%%%%%%%%
{
	\centering
	{\textbf{第 1 章 \indent 確率}} \\
}

{\textbf{\S 1.1 \indent 確率の定義と性質}}
\basic
\begin{qenumerate}
	\item{
		数字が 2 であるカードは 4 枚あるから
		\begin{align}
			P(A) = \frac{4}{52} = \red{\frac{1}{13}}.
		\end{align}
		数字が 8 以下であるカードは 32 枚あるから
		\begin{align}
			P(B) = \frac{32}{52} = \red{\frac{8}{13}}.
		\end{align}
		ハートの絵札は 3 枚 (ハートの Jack, Queen, King) あるから
		\begin{align}
			P(C) = \red{\frac{3}{52}}.
		\end{align}
	}
	\item{
		\begin{enumerate}
			\item{
				それぞれの硬貨が表である確率はそれぞれ $\dfrac{1}{2}$ であり, 求める確率はこれが同時に起こる確率だから
				\begin{align}
					\left(\frac{1}{2}\right)^{4} = \red{\frac{1}{16}}.
				\end{align}
			}
			\item{
				4 枚の硬貨のうち, 1 枚だけ表である場合は: 
				\begin{quote}
					表, 裏, 裏, 裏 \\
					裏, 表, 裏, 裏 \\
					裏, 裏, 表, 裏 \\
					裏, 裏, 裏, 表
				\end{quote}
				の 4 通りである.
				それぞれが起こる確率は
				\begin{align}
					\dfrac{1}{2}\cdot\left(\dfrac{1}{2}\right)^{3} = \dfrac{1}{16}
				\end{align}
				であるから
				\begin{align}
					4\cdot\dfrac{1}{16} = \red{\dfrac{1}{4}}.
				\end{align}
			}
		\end{enumerate}
	}
	\item{
		1 回のジャンケンの A の結果は, 勝ち, あいこ, 負け の 3 通りだから $\red{\dfrac{1}{3}}$.
	}
	\item{
		\begin{enumerate}
			\item{
				2 個のさいころの出る目の差は表のようになるから, $\dfrac{4}{36} = \red{\dfrac{1}{9}}$.
				\begin{table}[H]
					\centering
					\begin{tabular}{c|cccccc}
						  & 1 & 2 & 3 & 4 & 5 & 6 \\ \hline
						1 & 0 & 1 & 2 & 3 & {\textbf{4}} & 5 \\
						2 & 1 & 0 & 1 & 2 & 3 & {\textbf{4}} \\
						3 & 2 & 1 & 0 & 1 & 2 & 3 \\
						4 & 3 & 2 & 1 & 0 & 1 & 2 \\
						5 & {\textbf{4}} & 3 & 2 & 1 & 0 & 1 \\
						6 & 5 & {\textbf{4}} & 3 & 2 & 1 & 0
					\end{tabular}
				\end{table}
			}
			\item{
				2 個のさいころの出る目の和は表のようになるから, $\dfrac{3}{36} = \red{\dfrac{1}{12}}$.
				\begin{table}[H]
					\centering
					\begin{tabular}{c|cccccc}
						  & 1 & 2 & 3 & 4 & 5 & 6 \\ \hline
						1 & 2 & 3 & 4 & 5 & 6 & 7 \\
						2 & 3 & 4 & 5 & 6 & 7 & 8 \\
						3 & 4 & 5 & 6 & 7 & 8 & 9 \\
						4 & 5 & 6 & 7 & 8 & 9 & {\textbf{10}} \\
						5 & 6 & 7 & 8 & 9 & {\textbf{10}} & 11 \\
						6 & 7 & 8 & 9 & {\textbf{10}} & 11 & 12
					\end{tabular}
				\end{table}
			}
		\end{enumerate}
	}
	\item{
		袋から同時に 4 個の玉を取り出す取り出し方は $\combi{8}{4} = 70$ 通りである.
		\begin{enumerate}
			\item{
				取り出した 4 個の玉がすべて白玉である場合は $\combi{5}{4} = 5$ 通りだから $\dfrac{5}{70} = \red{\dfrac{1}{14}}$.
			}
			\item{
				取り出した 4 個の玉に白玉が 1 個だけ含まれる場合は
				\begin{align}
					\combi{5}{1}\cdot\combi{3}{3} = 5
				\end{align}
				より 5 通りだから $\dfrac{5}{70} = \red{\dfrac{1}{14}}$.
			}
			\item{
				取り出した 4 個の玉に白玉が 2 個だけ含まれる場合は
				\begin{align}
					\combi{5}{2}\cdot\combi{3}{2} = 10\cdot3 = 30
				\end{align}
				より 30 通りだから $\dfrac{30}{70} = \red{\dfrac{3}{7}}$.
			}
		\end{enumerate}
	}
	\item{
		\begin{enumerate}
			\item{
				800 以上の奇数ができるのは, 1 枚目に 8 のカードを引き, 3 枚目に 1, 3, 5, 7 のいずれかのカードを引いた場合である.
		
				1 枚目に 8 のカードを引き, 2 枚目に偶数, 3 枚目に奇数のカードを引く確率は
				\begin{align}
					\frac{1}{8}\cdot\frac{3}{7}\cdot\frac{4}{6} = \frac{1}{28}.
				\end{align}
				1 枚目に 8 のカードを引き, 2 枚目に奇数, 3 枚目に奇数のカードを引く確率は
				\begin{align}
					\frac{1}{8}\cdot\frac{4}{7}\cdot\frac{3}{6} = \frac{1}{28}.
				\end{align}
				すなわち, 求める確率はこれらの和をとって
				\begin{align}
					\frac{1}{28} + \frac{1}{28} = \red{\frac{1}{14}}.
				\end{align}
			}
			\item{
				200 以下の奇数ができるのは, 1 枚目に 1 のカードを引き, 3 枚目に 3, 5, 7 のいずれかのカードを引いた場合である.
		
				1 枚目に 1 のカードを引き, 2 枚目に偶数, 3 枚目に奇数のカードを引く確率は
				\begin{align}
					\frac{1}{8}\cdot\frac{4}{7}\cdot\frac{3}{6} = \frac{1}{28}.
				\end{align}
				1 枚目に 1 のカードを引き, 2 枚目に奇数, 3 枚目に奇数のカードを引く確率は
				\begin{align}
					\frac{1}{8}\cdot\frac{3}{7}\cdot\frac{2}{6} = \frac{1}{56}.
				\end{align}
				すなわち, 求める確率はこれらの和をとって
				\begin{align}
					\frac{1}{28} + \frac{1}{56} = \red{\frac{3}{56}}.
				\end{align}
			}
		\end{enumerate}
	}
	\item{
		4 個のさいころのうち, 3 個だけ同じ目になる場合は: 
		\begin{quote}
			同, 同, 同, 異 \\
			同, 同, 異, 同 \\
			同, 異, 同, 同 \\
			異, 同, 同, 同
		\end{quote}
		のような組み合わせの場合である.
		それぞれが起こる確率は
		\begin{align}
			6\cdot\left(\frac{1}{6}\right)^{3}\cdot\frac{5}{6} = \frac{5}{216}
		\end{align}
		であるから
		\begin{align}
			4\cdot\frac{5}{216} = \red{\frac{5}{54}}.
		\end{align}
	}
	\item{
		\begin{align}
			\frac{4623}{10000} = 0.4623 \simeq \red{0.46}.
		\end{align}
	}
	\item{
		\begin{enumerate}
			\item{
				\begin{description}
					\item[$A \cap B$]{
						大きいさいころの目が奇数かつ, 出る目の和が偶数である事象 $\Leftrightarrow$ \red{大きいさいころの目が奇数, 小さいさいころの目が奇数である事象}.
					}
					\item[$\overline{B}$]{
						\red{出る目の和が奇数である事象}.
					}
					\item[$\overline{A} \cap B$]{
						大きいさいころの目が偶数かつ, 出る目の和が偶数である事象 $\Leftrightarrow$ \red{大きいさいころの目が偶数, 小さいさいころの目が偶数である事象}.
					}
					\item[$A \cup \overline{B}$]{
						\red{大きいさいころの目が奇数または出る目の和が奇数}.
					}
				\end{description}
			}
			\item{
				$(A \cup B) \cap C = \emptyset$ より $C = \overline{A \cup B} = \overline{A} \cap \overline{B}$ とすればよいから, 大きいさいころの目が偶数かつ, 出る目の和が奇数である事象 $\Leftrightarrow$ \red{大きいさいころの目が偶数, 小さいさいころの目が奇数}.
			}
		\end{enumerate}
	}
	\item{
		\begin{enumerate}
			\item{
				トランプ 52 枚のうち, 奇数のカードは 28 枚あるから
				\begin{align}
					P(A) = \frac{28}{52}\cdot\frac{27}{51} = \red{\frac{63}{221}}.
				\end{align}
			}
			\item{
				カードの数の和が 9 となるのは, 2 枚のカードの数が $(1, 8)$, $(2, 7)$, $(3, 6)$, $(4, 5)$, $(5, 4)$, $(6, 3)$, $(7, 2)$, $(8, 1)$ の 8 通りであり, 2 枚のカードのスートの組み合わせが $4^{2} = 16$ 通りであるから
				\begin{align}
					P(B) = \frac{8\cdot16}{52\cdot51} = \red{\frac{32}{663}}.
				\end{align}
			}
			\item{
				奇数の和は偶数であり, 9 は奇数だから $A$ と $B$ は互いに排反であるから
				\begin{align}
					P(A\cup B) = P(A) + P(B) = \frac{63}{221} + \frac{32}{663} = \red{\frac{1}{3}}.
				\end{align}
			}
		\end{enumerate}
	}
	\item{
		\begin{enumerate}
			\item{
				トランプ 52 枚のうち, 絵札のカードは 12 枚あるから
				\begin{align}
					\frac{12}{52}\cdot\frac{11}{51} = \red{\frac{11}{221}}.
				\end{align}
			}
			\item{
				``少なくとも 1 枚は絵札でない'' 事象の余事象は ``2 枚とも絵札である'' 事象であるから, (1) より
				\begin{align}
					1 - \frac{11}{221} = \red{\frac{210}{221}}.
				\end{align}
			}
		\end{enumerate}
	}
	\item{
		$A$, $B$ の確率はそれぞれ $P(A) = \dfrac{5}{10}$, $P(B) = \dfrac{4}{10}$ である.
		\begin{enumerate}
			\item{
				10 枚のカードのうち, 奇数かつ素数であるカードは 3, 5, 7 だから, $A\cap B$ の確率は $P(A\cap B) = \dfrac{3}{10}$ である.
				すなわち
				\begin{align}
					P(A\cup B) &= P(A) + P(B) - P(A\cap B) \\
						&= \frac{5}{10} + \frac{4}{10} - \frac{3}{10} = \red{\frac{3}{5}}.
				\end{align}
			}
			\item{
				10 枚のカードのうち, 偶数かつ素数であるカードは 2 のみだから, $\overline{A}\cap B$ の確率は $P\left(\overline{A}\cap B\right) = \dfrac{1}{10}$ である.
				すなわち
				\begin{align}
					P\left(\overline{A}\cup B\right) &= P\left(\overline{A}\right) + P(B) - P\left(\overline{A}\cap B\right) \\
						&= \left(1 - \frac{5}{10}\right) + \frac{4}{10} - \frac{1}{10} = \red{\frac{4}{5}}.
				\end{align}
			}
			\item{
				10 枚のカードのうち, 奇数かつ素数でないカードは 1, 9 だから, $A\cap \overline{B}$ の確率は $P\left(A\cap\overline{B}\right) = \dfrac{2}{10}$ である.
				すなわち
				\begin{align}
					P\left(A\cup\overline{B}\right) &= P(A) + P\left(\overline{B}\right) - P\left(A\cup\overline{B}\right) \\
						&= \frac{5}{10} + \left(1 - \frac{4}{10}\right) - \frac{2}{10} = \red{\frac{9}{10}}.
				\end{align}
			}
		\end{enumerate}
	}
	\item{
		\begin{enumerate}
			\item{
				3 枚の硬貨を投げ, 1 枚も表が出ない確率は 
				\begin{align}
					\left(\dfrac{1}{2}\right)^{3} = \frac{1}{8}, 
				\end{align}
				1 枚だけ表が出る確率は
				\begin{align}
					3\cdot\left(\dfrac{1}{2}\right)^{3} = \dfrac{3}{8}, 
				\end{align}
				2 枚だけ表が出る確率は
				\begin{align}
					3\cdot\left(\dfrac{1}{2}\right)^{3} = \dfrac{3}{8}, 
				\end{align}
				3 枚表が出る確率は
				\begin{align}
					\left(\dfrac{1}{2}\right)^{3} = \dfrac{1}{8}
				\end{align}
				である.
				すなわち
				\begin{align}
					0\cdot\frac{1}{8} + 1\cdot\frac{3}{8} + 2\cdot\frac{3}{8} + 3\cdot\frac{1}{8} = \red{\frac{3}{2}}.
				\end{align}
			}
			\item{
				2 個のさいころの出る目の差は表のようになる.
				\begin{table}[H]
					\centering
					\begin{tabular}{c|cccccc}
						  & 1 & 2 & 3 & 4 & 5 & 6 \\ \hline
						1 & 0 & 1 & 2 & 3 & 4 & 5 \\
						2 & 1 & 0 & 1 & 2 & 3 & 4 \\
						3 & 2 & 1 & 0 & 1 & 2 & 3 \\
						4 & 3 & 2 & 1 & 0 & 1 & 2 \\
						5 & 4 & 3 & 2 & 1 & 0 & 1 \\
						6 & 5 & 4 & 3 & 2 & 1 & 0
					\end{tabular}
				\end{table}
				すなわち
				\begin{align}
					& 0\cdot\frac{6}{36} + 1\cdot\frac{10}{36} + 2\cdot\frac{8}{36} + 3\cdot\frac{6}{36} + 4\cdot\frac{4}{36} + 5\cdot\frac{2}{36} \\
						=& \red{\frac{35}{18}}.
				\end{align}
			}
		\end{enumerate}
	}
	\item{
		500 円の当たりくじを引く確率は $\dfrac{1}{12}$, 200 円の当たりくじを引く確率は $\dfrac{2}{12}$, はずれくじを引く確率は $\dfrac{9}{12}$ だから
		\begin{align}
			500\cdot\frac{1}{12} + 200\cdot\frac{2}{12} + 0\cdot\frac{9}{12} = 75
		\end{align}
		より, \red{75 円}.
	}
	\item{
		取り出した 2 枚のカードから $x$ の値を計算すると表のようになる.
		\begin{table}[H]
			\centering
			\begin{tabular}{c|cccccccccc}
				   & 1 & 2 & 3 & 4 & 5 & 6 & 7 & 8 & 9 & 10 \\ \hline
				 1 & - & 1 & 2 & 3 & 4 & 5 & 6 & 7 & 8 &  9 \\
				 2 & 1 & - & 1 & 2 & 3 & 4 & 5 & 6 & 7 &  8 \\
				 3 & 2 & 1 & - & 1 & 2 & 3 & 4 & 5 & 6 &  7 \\
				 4 & 3 & 2 & 1 & - & 1 & 2 & 3 & 4 & 5 &  6 \\
				 5 & 4 & 3 & 2 & 1 & - & 1 & 2 & 3 & 4 &  5 \\
				 6 & 5 & 4 & 3 & 2 & 1 & - & 1 & 2 & 3 &  4 \\
				 7 & 6 & 5 & 4 & 3 & 2 & 1 & - & 1 & 2 &  3 \\
				 8 & 7 & 6 & 5 & 4 & 3 & 2 & 1 & - & 1 &  2 \\
				 9 & 8 & 7 & 6 & 5 & 4 & 3 & 2 & 1 & - &  1 \\
				10 & 9 & 8 & 7 & 6 & 5 & 4 & 3 & 2 & 1 &  - \\
			\end{tabular}
		\end{table}
		\begin{enumerate}
			\item{
				表より, $x = 2$ となる確率は
				\begin{align}
					\frac{16}{90} = \red{\frac{8}{45}}.
				\end{align}
			}
			\item{
				表より, $x = 6$ となる確率は
				\begin{align}
					\frac{8}{90} = \red{\frac{4}{45}}.
				\end{align}
			}
			\item{
				表より, $x$ の期待値は
				\begin{align}
					&1\cdot\frac{18}{90} + 2\cdot\frac{16}{90} + 3\cdot\frac{14}{90} + 4\cdot\frac{12}{90} + 5\cdot\frac{10}{90} \\
						&\textbf{\indent\indent} + 6\cdot\frac{8}{90} + 7\cdot\frac{6}{90} + 8\cdot\frac{4}{90} + 9\cdot\frac{2}{90} \\
						=& \red{\frac{11}{3}}.
				\end{align}
			}
		\end{enumerate}
	}
\end{qenumerate}

\vspace{\baselineskip}
\check
\begin{qenumerate}
	\item{
		\begin{enumerate}
			\item{
				偶数の玉は 2, 4, 6, 8 の 4 個だから $\red{\dfrac{4}{9}}$.
			}
			\item{
				3 の倍数の玉は 3, 6, 9 の 3 個だから $\dfrac{3}{9} = \red{\dfrac{1}{3}}$.
			}
		\end{enumerate}
	}
	\item{
		\begin{enumerate}
			\item{
				出る目の積が 12 となるのは $(2, 6)$, $(3, 4)$, $(4, 3)$, $(6, 2)$ の 4 通りだから $\dfrac{4}{36} = \red{\dfrac{1}{9}}$.
			}
			\item{
				出る目の和が 10 となるのは $(4, 6)$, $(5, 5)$, $(5, 6)$, $(6, 4)$, $(6, 5)$, $(6, 6)$ の 6 通りだから $\dfrac{6}{36} = \red{\dfrac{1}{6}}$.
			}
		\end{enumerate}
	}
	\item{
		\begin{enumerate}
			\item{
				5 本の当たりくじのうち 3 本を同時に引ければよいから
				\begin{align}
					\frac{5}{20}\cdot\frac{4}{19}\cdot\frac{3}{18} = \red{\frac{1}{114}}.
				\end{align}
			}
			\item{
				2 本が当たり 1 本が外れる場合は $\dfrac{\permu{3}{3}}{\permu{2}{2}} = 3$ 通りあり, それぞれが起こる確率は
				\begin{align}
					\frac{5}{20}\cdot\frac{4}{19}\cdot\frac{15}{18} = \frac{5}{114}
				\end{align}
				であるから, 求める確率は
				\begin{align}
					3\cdot\frac{5}{114} = \red{\frac{5}{38}}.
				\end{align}
			}
		\end{enumerate}
	}
	\item{
		\begin{enumerate}
			\item[(i)]{
				``偶-偶-偶'' の順にカードを取り出す確率は
				\begin{align}
					\frac{4}{9}\cdot\frac{3}{8}\cdot\frac{2}{7} = \frac{24}{504}.
				\end{align}
			}
			\item[(ii)]{
				``偶-奇-偶'' または ``奇-偶-偶'' の順にカードを取り出す確率は
				\begin{align}
					\frac{4}{9}\cdot\frac{5}{8}\cdot\frac{3}{7} + \frac{5}{9}\cdot\frac{4}{8}\cdot\frac{3}{7} = \frac{120}{504}.
				\end{align}
			}
			\item[(iii)]{
				``奇-奇-偶'' の順にカードを取り出す確率は
				\begin{align}
					\frac{5}{9}\cdot\frac{4}{8}\cdot\frac{4}{7} = \frac{80}{504}.
				\end{align}
			}
		\end{enumerate}
		以上より
		\begin{align}
			\frac{24}{504} + \frac{120}{504} + \frac{80}{504} = \red{\frac{4}{9}}.
		\end{align}
	}
	\item{
		``男-男'' が選ばれる確率は
		\begin{align}
			\frac{8}{14}\cdot\frac{7}{13} = \frac{56}{182}
		\end{align}
		であり, ``女-女'' が選ばれる確率は
		\begin{align}
			\frac{6}{14}\cdot\frac{5}{13} = \frac{30}{182}
		\end{align}
		であるから
		\begin{align}
			\frac{56}{182} + \frac{30}{182} = \red{\frac{43}{91}}.
		\end{align}
	}
	\item{
		\begin{enumerate}
			\item{
				\begin{enumerate}
					\item{
						赤玉が 0 個である確率は
						\begin{align}
							\frac{5}{12}\cdot\frac{4}{11}\cdot\frac{3}{10}\cdot\frac{2}{9} = \frac{120}{11880}.
						\end{align}
					}
					\item{
						赤玉が 1 個である場合は $\dfrac{\permu{4}{4}}{\permu{3}{3}} = 4$ 通りあり, それぞれが起こる確率は
						\begin{align}
							\frac{7}{12}\cdot\frac{5}{11}\cdot\frac{4}{10}\cdot\frac{3}{9} = \frac{420}{11880}
						\end{align}
						であるから, 赤玉が 1 個である確率は
						\begin{align}
							4\cdot\frac{420}{11880} = \frac{1680}{11880}.
						\end{align}
					}
				\end{enumerate}
				以上より, 求める確率は
				\begin{align}
					\frac{120}{11880} + \frac{1680}{11880} = \red{\frac{5}{33}}.
				\end{align}
			}
			\item{
				``少なくとも 1 個は赤玉である'' 事象の余事象は ``赤玉が 0 個である'' 事象であるから, (1)-(i) より
				\begin{align}
					1 - \frac{120}{11880} = \red{\frac{98}{99}}.
				\end{align}
			}
		\end{enumerate}
	}
	\item{
		$A$, $B$ の確率はそれぞれ $P(A) = \dfrac{50}{100}$, $P(B) = \dfrac{33}{100}$ である.
		\begin{enumerate}
			\item{
				カードの数字が 6 の倍数である確率だから
				\begin{align}
					P(A\cap B) = \frac{16}{100} = \red{\frac{4}{25}}.
				\end{align}
			}
			\item{
				(1) より
				\begin{align}
					P(A\cup B) &= P(A) + P(B) - P(A\cap B) \\
						&= \frac{50}{100} + \frac{33}{100} - \frac{16}{100} = \red{\frac{67}{100}}.
				\end{align}
			}
			\item{
				$A\cap\overline{B}$ は 2 倍数のうち 6 の倍数を除いた事象だから, その確率は
				\begin{align}
					P\left(A\cap\overline{B}\right) = \frac{50}{100} - \frac{16}{100} = \frac{34}{100}.
				\end{align}
				よって
				\begin{align}
					P\left(A\cup\overline{B}\right) &= P(A) + P\left(\overline{B}\right) - P\left(A\cap\overline{B}\right) \\
						&= \frac{50}{100} + \left(1 - \frac{33}{100}\right) - \frac{34}{100} = \red{\frac{83}{100}}.
				\end{align}
			}
		\end{enumerate}
	}
	\item{
		\begin{enumerate}
			\item[(i)]{
				赤玉をちょうど 1 個取り出す取り出し方は $\dfrac{\permu{3}{3}}{\permu{2}{2}} = 3$ 通りあり, それぞれの確率は
				\begin{align}
					\frac{3}{10}\cdot\frac{7}{9}\cdot\frac{6}{8} = \frac{126}{720}
				\end{align}
				だから, 赤玉をちょうど 1 個取り出す確率は
				\begin{align}
					3\cdot\frac{126}{720} = \frac{378}{720}.
				\end{align}
			}
			\item[(ii)]{
				赤玉をちょうど 2 個取り出す取り出し方は $\dfrac{\permu{3}{3}}{\permu{2}{2}} = 3$ 通りあり, それぞれの確率は
				\begin{align}
					\frac{3}{10}\cdot\frac{2}{9}\cdot\frac{7}{8} = \frac{42}{720}
				\end{align}
				だから, 赤玉をちょうど 2 個取り出す確率は
				\begin{align}
					3\cdot\frac{42}{720} = \frac{126}{720}.
				\end{align}
			}
			\item[(iii)]{
				赤玉を 3 個取り出す確率は
				\begin{align}
					\frac{3}{10}\cdot\frac{2}{9}\cdot\frac{1}{8} = \frac{6}{720}.
				\end{align}
			}
		\end{enumerate}
		以上より, 賞金額の期待値は
		\begin{align}
			100\cdot\frac{378}{720} + 200\cdot\frac{126}{720} + 300\cdot\frac{6}{720} = 90
		\end{align}
		より, \red{90 円}.
	}
	\item{
		2 個のさいころの出た目から $x$ の値を計算すると表のようになる.
		\begin{table}[H]
			\centering
			\begin{tabular}{c|cccccc}
				& 1 & 2 & 3 & 4 & 5 & 6 \\ \hline
				1 & 2 & 3 & 0 & 1 & 2 & 3 \\
				2 & 3 & 0 & 1 & 2 & 3 & 0 \\
				3 & 0 & 1 & 2 & 3 & 0 & 1 \\
				4 & 1 & 2 & 3 & 0 & 1 & 2 \\
				5 & 2 & 3 & 0 & 1 & 2 & 3 \\
				6 & 3 & 0 & 1 & 2 & 3 & 0
			\end{tabular}
		\end{table}
		\begin{enumerate}
			\item{
				表より, $x = 0$ となる確率は
				\begin{align}
					\frac{9}{36} = \red{\frac{1}{4}}.
				\end{align}
			}
			\item{
				表より, $x$ の期待値は
				\begin{align}
					0\cdot\frac{9}{36} + 1\cdot\frac{8}{36} + 2\cdot\frac{9}{36} + 3\cdot\frac{10}{36} = \red{\frac{9}{14}}.
				\end{align}
				}
		\end{enumerate}
	}
\end{qenumerate}

\vspace{\baselineskip}
\stepup
\begin{qenumerate}
	\item{
		\begin{proof}
			仮定より
			\begin{align}
				\sqrt{P(A)P(B)}>\sqrt{\frac{1}{4}}=\frac{1}{2}.
			\end{align}
			相加平均と相乗平均の関係より
			\begin{align}
				\frac{P(A) + P(B)}{2}\geq\frac{1}{2}
			\end{align}
			であるから
			\begin{align}
				P(A) + P(B) \geq 1.
			\end{align}
			したがって, $A$, $B$ は互いに排反でない.
		\end{proof}
	}
	\item{
		\begin{enumerate}
			\item{
				$\overline{A}$ は ``3 回とも裏である'' 事象であり, これは $B$ に含まれるから, $A\cup B$ は全事象である.
				したがって $P(A\cup B) = \red{1}$.
			}
			\item{
				$B$ が起こるとき, 3 回目が裏となるから, $A$ が起こるのは 1 回目か 2 回目に少なくとも 1 回表が出る場合である.
				``1 回目か 2 回目に少なくとも 1 回表が出る'' 事象を $A'$ とすると, その余事象 $\overline{A'}$ は, ``1 回目と 2 回目ともに裏が出る'' となり, この確率は
				\begin{align}
					P\left(\overline{A'}\right) = (1-p)^{2}
				\end{align}
				である.
				これより, $A'$ の起こる確率は
				\begin{align}
					P\left(A'\right) = 1 - P\left(\overline{A'}\right) = 1 - (1 - p)^{2} = p(2 - p)
				\end{align}
				だから
				\begin{align}
					P(A\cap B) = P\left(A'\right)\cdot(1 - p) = \red{p(1-p)(2-p)}.
				\end{align}
			}
		\end{enumerate}
	}
	\item{
		\begin{enumerate}
			\item{
				$n$ 人全員の誕生日が異なる確率を考える.
				1 人目の誕生日は 365 通りあり, 2 人目の誕生日は 1 人目の誕生日以外から選ぶので 364 通りある.
				これを $n$ 人目まで繰り返すと, $n$ 人全員の誕生日が異なる確率は
				\begin{align}
					\underbrace{\frac{365}{365}\cdot\frac{364}{365}\cdot\cdots\cdot\frac{365 - n + 1}{365}}_{n} = \frac{\frac{365!}{(365-n)!}}{365^{n}} = \frac{\permu{365}{n}}{365^{n}}
				\end{align}
				となるから
				\begin{align}
					p_{n} = \red{1 - \frac{\permu{365}{n}}{365^{n}}} = \red{1 - \frac{365!}{365^{n}\cdot(365 - n)!}}
				\end{align}
			}
			\item{
				\begin{align}
					p_{5} = 1 - \frac{365}{365}\cdot\frac{364}{365}\cdot\frac{363}{365}\cdot\frac{362}{365}\cdot\frac{361}{365} \simeq \red{0.0271}.
				\end{align}
			}
		\end{enumerate}
	}
	\item{
		$n$ 人がじゃんけんをして勝敗が決まる確率を考える.
		勝敗が決まるのは, 出された手が 2 種類のみのときであり, この手の選び方は $\combi{3}{2} = 3$ 通りある.
		それぞれの確率は, $n$ 人が 2 種類の手を出す出し方の $2^{n}$ 通りから, 2 種類のどちらかの手で全員の手が揃う出し方を引けばよいから
		\begin{align}
			\frac{2^{n}}{3^{n}} - \frac{1}{3^{n}} - \frac{1}{3^{n}} = \frac{2^{n} - 2}{3^{n}}.
		\end{align}
		すなわち, $n$ 人がじゃんけんをして勝敗が決まる確率は
		\begin{align}
			3\cdot\frac{2^{n} - 2}{3^{n}} = \frac{3\cdot ^{n} - 6}{3^{n}}.
		\end{align}
		したがって, 求める確率は
		\begin{align}
			p_{n} = 1 - \frac{3\cdot 2^{n} - 6}{3^{n}} = \red{\frac{3^{n} - 3\cdot 2^{n} + 6}{3^{n}}}.
		\end{align}
	}
	\item{
		\begin{enumerate}
			\item{
				1 枚目に偶数, 2枚目に偶数のカードを引く確率は
				\begin{align}
					\frac{4}{9}\cdot\frac{3}{8} = \frac{12}{72}
				\end{align}
				であり, 1 枚目に奇数, 2 枚目に偶数のカードを引く確率は
				\begin{align}
					\frac{5}{9}\cdot\frac{4}{8} = \frac{20}{72}
				\end{align}
				であるから, 求める確率は
				\begin{align}
					\frac{12}{72} + \frac{20}{72} = \red{\frac{4}{9}}.
				\end{align}
			}
			\item{
				2 けたの数字が 3 の倍数となる条件は, 各けたの数字の和が 3 の倍数となることである.
				1 枚目に 3 の倍数のカードを引いた場合, 和が 3 の倍数になるのは 2 枚目にも 3 の倍数のカードを引いた場合である.
				この場合の確率は
				\begin{align}
					\frac{3}{9}\cdot\frac{2}{8} = \frac{6}{72}.
				\end{align}
				1 枚目に 3 の倍数以外のカードを引いた場合, 和を 3 の倍数にするために必要な 2 枚目のカードは 3 通りあり, この場合の確率は
				\begin{align}
					\frac{6}{9}\cdot\frac{3}{8} = \frac{18}{72}.
				\end{align}
				すなわち, 求める確率は
				\begin{align}
					\frac{6}{72} + \frac{18}{72} = \frac{24}{72} = \red{\frac{1}{3}}.
				\end{align}
			}
			\item{
				2 けたの数字の作られ方を考えると, 十の位, 一の位にはそれぞれ 1 から 9 の数字が 8 回ずつ出現するから, 2 けたの数字の総和は
				\begin{align}
					10\cdot 8\cdot(1 + \cdots + 9) + 1\cdot 8\cdot(1 + \cdots + 9) = 3960.
				\end{align}
				それぞれの数字の出現確率は等しく $\dfrac{1}{72}$ だから, 求める期待値は
				\begin{align}
					\frac{3960}{72} = \red{55}.
				\end{align}
			}
		\end{enumerate}
	}
	\item{
		8 個の玉を円形に並べる並べ方は $(8 - 1)! = 7!$ 通りである.
		\begin{enumerate}
			\item{
				円形に並べた 5 個の白玉の間に赤玉を並べることを考える.
				5 個の白玉を円形に並べる並べ方は $(5 - 1)! = 4!$ 通りあり, 5 箇所の白玉の間から 3 箇所選んで赤玉を並べればよいから, 求める確率は
				\begin{align}
					\frac{4!\cdot \permu{5}{3}}{7!} = \frac{4!\cdot (5\cdot 4\cdot 3)}{7!} = \red{\frac{2}{7}}.
				\end{align}
			}
			\item{
				3 個の赤玉をまとめて 1 個と捉えて, 6 個の玉を円形に並べることを考える.
				6 個の玉を円形に並べる並べ方は $(6 - 1)! = 5!$ 通りあり, それぞれについてまとめた 3 個の赤玉の並べ方が $\permu{3}{3} = 3!$ 通りあるから, 求める確率は
				\begin{align}
					\frac{5!\cdot 3!}{7!} = \red{\frac{1}{7}}.
				\end{align}
			}
		\end{enumerate}
	}
	\item{
		8 名が丸く並んで輪を作る作り方は $(8 - 1)! = 7!$ 通りである.
		\begin{enumerate}
			\item{
				4 組の親子が丸く並ぶ並び方は $(4 - 1)! = 3!$ 通りである.
				それぞれについて, 親子内での並び方が 2 通りあり, 親子は 4 組あるから, 求める確率は
				\begin{align}
					\frac{3!\cdot 2^{4}}{7!} = \red{\frac{2}{105}}.
				\end{align}
			}
			\item{
				(1) の余事象だから
				\begin{align}
					1 - \frac{2}{105} = \red{\frac{103}{105}}.
				\end{align}
			}
			\item{
				丸く並んだ 4 名の大人の間に子どもを並べることを考える.
				4 名の大人を丸く並べる並べ方は $(4 - 1)! = 3!$ 通りあり, その間に 4 名の子どもを並べればよいから
				\begin{align}
					\frac{3!\cdot \permu{4}{4}}{7!} = \frac{3!\cdot 4!}{7!} = \red{\frac{1}{35}}.
				\end{align}
			}
			\item{
				(1) の事象を $A$, (3) の事象を $B$ とすると, 考える事象は $\overline{A}\cup\overline{B} = \overline{A\cap B}$ であり, $A\cap B$ は ``すべての親子が隣り合い, 大人と子どもが交互になっている'' ことを表す.
				4 名の大人を丸く並べる並べ方は $3!$ 通りあり, それぞれの子どもの並べ方は親の右側か左側の 2 通りだから
				\begin{align}
					P(A\cap B) = \frac{3!\cdot 2}{7!} = \frac{1}{420}.
				\end{align}
				すなわち, 求める確率は
				\begin{align}
					P\left(\overline{A}\cup \overline{B}\right) &= P\left(\overline{A\cap B}\right) = 1 - P(A\cap B) \\
						&= 1 - \frac{1}{420} = \red{\frac{419}{420}}.
				\end{align}
			}
		\end{enumerate}
	}
	\item{
		\begin{enumerate}
			\item{
				2 点を選んでできる直線の総数は $\combi{8}{2} = 28$ 本である.
				このうち, 円の直径となるのは 4 本だから $\dfrac{4}{28} = \red{\dfrac{1}{7}}$.
			}
			\item{
				3 点を選んでできる三角形の総数は $\combi{8}{3} = 56$ 個である.
				3 点を選んで直角三角形となるのは, 三角形の 1 辺が円の直径を含む場合である\footnote{円周角の定理より, 円周上に直角が存在するためにはこれが直径に対する円周角になっていることが必要である.}.
				直角三角形となるものは, 1 本の対角線に対して 6 通りあるから, 求める確率は (1) より
				\begin{align}
					\frac{4\cdot 6}{56} = \red{\frac{3}{7}}.
				\end{align}
			}
			\item{
				ある 1 点に対し, これを頂角とする二等辺三角形は 3 個あり, このうち 1 個は直角三角形になるから, 直角三角形でない二等辺三角形は $8\cdot (3 - 1) = 16$ 個である.
				すなわち, 求める確率は (2) より
				\begin{align}
					1 - \frac{24 + 16}{56} = \red{\frac{2}{7}}.
				\end{align}
			}
			\item{
				4 点を選んでできる四角形の総数は $\combi{8}{4} = 70$ 個である.
				このうち, 正方形となるのは 2 個だから $\dfrac{2}{70} = \red{\dfrac{1}{35}}$.
			}
			\item{
				隣り合う 2 点を結んでできる辺に対し, これを底辺とする台形は 3 個あり, このうち 1 個は長方形になるから, 長方形でない台形は $8\cdot (3 - 1) = 16$ 個である.
				隣り合う 2 点を結んでできる辺は 8 本あり, これらの中から向かい合わせの 2 辺を選ぶと長方形ができるから, 長方形は $\dfrac{8}{2} = 4$ 個である.
				1 点を挟んだ 2 点を結んでできる辺に対し, これを底辺とする台形は 2 個あり, このうち 1 個は正方形になるから, 正方形でない台形は $8\cdot (2 - 1) = 8$ 個である.
				(4) より, 正方形は 2 個だから, 求める確率は
				\begin{align}
					\frac{16 + 4 + 8 + 2}{70} = \red{\frac{3}{7}}.
				\end{align}
			}
		\end{enumerate}
	}
	\item{
		3 点を選んでできる三角形の総数は $\combi{12}{3} = 220$ 個である.
		また, 正十二角形の各頂点は同一円周上にある.
		\begin{enumerate}
			\item{
				正三角形となる 3 点の選び方は 4 通りだから $\dfrac{4}{220} = \red{\dfrac{1}{55}}$.
			}
			\item{
				3 点を選んで直角三角形となるのは, 三角形の 1 辺が円の直径を含む場合である.
				2 点を選んでできる直線のうち, 円の直径になるのは 6 通りあり, それぞれの辺に対して直角三角形となる頂点の選び方は 10 通りあるから, 求める確率は
				\begin{align}
					\frac{6\cdot 10}{220} = \red{\frac{3}{11}}.
				\end{align}
			}
			\item{
				ある 1 点に対し, これを頂点とする二等辺三角形は 5 個あり, このうち 1 個は正三角形だから, 正三角形でない二等辺三角形は $12\cdot (5 - 1) = 48$ 個である.
				(1) より, 正三角形は 4 個だから, 求める確率は
				\begin{align}
					\frac{48 + 4}{220} = \red{\frac{13}{55}}.
				\end{align}
			}
		\end{enumerate}
	}
	\item{
		引いた 2 枚のカードを $(i, j)$ と表す.
		\begin{enumerate}
			\item{
				$x = k$ となるのは, $(1, k - 1)$, $(2, k - 2)$, $\dots$, $(k - 1, 1)$ の $k - 1$ 通りだから, 求める確率は $\red{\dfrac{k - 1}{n^{2}}}$.
			}
			\item{
				$x = n + k$ となるのは, $(k, n)$, $(k + 1, n - 1)$, $\dots$, $(n, k)$ の $n - k + 1$ 通りだから, 求める確率は $\red{\dfrac{n - k + 1}{n^{2}}}$.
			}
		\end{enumerate}
	}
\end{qenumerate}

\vspace{\baselineskip}
{\textbf{\S 1.2 \indent いろいろな確率}}
\basic
\begin{qenumerate}
	\item{
		$P(A) = \dfrac{6}{52} = \dfrac{3}{26}$, $P(B) = \dfrac{12}{52} = \dfrac{6}{26}$ である.
		また, $A\cap B$ は ``スペードの素数かつ絵札'' すなわち, ``スペードの Jack または スペードの King'' という事象だから
		\begin{align}
			P(A\cap B) = \frac{2}{52} = \frac{1}{26}.
		\end{align}
		したがって
		\begin{align}
			P_{A}(B) &= \frac{P(A\cap B)}{P(A)} = \frac{\frac{1}{26}}{\frac{3}{26}} = \red{\frac{1}{3}}, \\
			P_{B}(A) &= \frac{P(A\cap B)}{P(B)} = \frac{\frac{1}{26}}{\frac{6}{26}} = \red{\frac{1}{6}}.
		\end{align}
	}
	\item{
		\begin{enumerate}
			\item{
				大きいさいころの出る目が偶数であるのは 2, 4, 6 の 3 通りだから
				\begin{align}
					P(A) = \frac{3}{6} = \red{\frac{1}{2}}.
				\end{align}
			}
			\item{
				大きいさいころの出る目が偶数であるとき, それぞれについて和を 7 にするための小さいさいころの出る目は 1 通りだから $P_{A}(B) = \red{\dfrac{1}{6}}$.
			}
			\item{
				確率の乗法定理より
				\begin{align}
					P(A\cap B) = P(A)P_{A}(B) = \frac{1}{2}\cdot\frac{1}{6} = \red{\frac{1}{12}}.
				\end{align}
			}
		\end{enumerate}
	}
	\item{
		\begin{enumerate}
			\item{
				野球観戦が好きな会員のうち, 30\% がサッカー観戦は好きではない会員だから
				\begin{align}
					\frac{40}{100}\cdot\frac{30}{100} = \red{\frac{3}{25}}.
				\end{align}
			}
			\item{
				野球観戦が好きな会員の 70\% がサッカー観戦も好きで, さらにその 80\% がテニス観戦も好きな会員だから
				\begin{align}
					\frac{40}{100}\cdot\frac{70}{100}\cdot\frac{80}{100} = \red{\frac{28}{125}}.
				\end{align}
			}
			\item{
				野球観戦もサッカー観戦も好きな会員のうち, 20\% がテニス観戦は好きではない会員だから
				\begin{align}
					\frac{40}{100}\cdot\frac{70}{100}\cdot\frac{20}{100} = \red{\frac{7}{125}}.
				\end{align}
			}
		\end{enumerate}
	}
	\item{
		\begin{enumerate}
			\item{
				A が引く時点では 20 本のくじの中に当たりくじが 4 本あるから $\dfrac{4}{20} = \red{\dfrac{1}{5}}$.
			}
			\item{
				A が当たっていた場合, B が引く時点では 19 本のくじの中に当たりくじが 3 本あるから
				\begin{align}
					\frac{1}{5}\cdot\frac{3}{19} = \frac{3}{95}.
				\end{align}
				A が当たっていない場合, B が引く時点では 19 ほんのくじの中に当たりくじが 4 本あるから
				\begin{align}
					\frac{4}{5}\cdot\frac{4}{19} = \frac{16}{95}.
				\end{align}
				すなわち, B が当たる確率は
				\begin{align}
					\frac{3}{95} + \frac{16}{95} = \red{\frac{1}{5}}.
				\end{align}
			}
			\item{
				A が当たった上で B も当たる確率は
				\begin{align}
					\frac{1}{5}\cdot\frac{3}{19} = \frac{3}{95}.
				\end{align}
				この上で C も当たる確率は
				\begin{align}
					\frac{3}{95}\cdot\frac{2}{18} = \red{\frac{1}{285}}.
				\end{align}
			}
			\item{
				A が当たった上で B が外れる確率は
				\begin{align}
					\frac{1}{5}\cdot\frac{16}{19} = \frac{16}{95}.
				\end{align}
				この上で, C が当たる確率は
				\begin{align}
					\frac{16}{95}\cdot\frac{3}{18} = \red{\frac{8}{285}}.
				\end{align}
			}
			\item{
				A, B がともにはずれて C が当たる確率は
				\begin{align}
					\frac{16}{20}\cdot\frac{15}{19}\cdot\frac{4}{18} = \frac{24}{171}.
				\end{align}
				したがって, (3), (4) より
				\begin{align}
					\frac{1}{285} + 2\cdot\frac{8}{285} + \frac{24}{171} = \red{\frac{1}{5}}.
				\end{align}
			}
		\end{enumerate}
	}
	\item{
		\begin{enumerate}
			\item{
				\begin{enumerate}
					\item{
						1 から 600 までの整数から選ぶ場合: \\
						$A$ の起こる確率 $P(A)$ は
						\begin{align}
							P(A) = \frac{200}{600} = \frac{1}{3}, 
						\end{align}
						$B$ の起こる確率 $P(B)$ は
						\begin{align}
							P(B) = \frac{120}{600} = \frac{1}{5}
						\end{align}
						である.
						また, $A\cap B$ の起こる確率 $P(A\cap B)$ は
						\begin{align}
							P(A\cap B) = \frac{40}{600} = \frac{1}{15}.
						\end{align}
						である.
						したがって
						\begin{align}
							P(A\cap B) = P(A)P(B).
						\end{align}
						すなわち, $A$, $B$ は\red{互いに独立である}.
					}
					\item{
						1 から 400 までの整数から選ぶ場合: \\
						$A$ の起こる確率 $P(A)$ は
						\begin{align}
							P(A) = \frac{133}{400}, 
						\end{align}
						$B$ の起こる確率 $P(B)$ は
						\begin{align}
							P(B) = \frac{80}{400} = \frac{1}{5}
						\end{align}
						である.
						また, $A\cap B$ の起こる確率 $P(A\cap B)$ は
						\begin{align}
							P(A\cap B) = \frac{26}{400} = \frac{13}{200}.
						\end{align}
						である.
						したがって
						\begin{align}
							P(A\cap B) \neq P(A)P(B).
						\end{align}
						すなわち, $A$, $B$ は\red{互いに独立でない}.
					}
				\end{enumerate}
			}
			\item{
				$A$, $B$, $C$ のそれぞれが起こる確率 $P(A)$, $P(B)$, $P(C)$ は
				\begin{align}
					P(A) &= \frac{3}{6} = \frac{1}{2}, \\
					P(B) &= \frac{3}{6} = \frac{1}{2}, \\
					P(C) &= \frac{2}{6} = \frac{1}{3}.
				\end{align}
				また, $A\cap B$, $B\cap C$, $C\cap A$ のそれぞれが起こる確率 $P(A\cap B)$, $P(B\cap C)$, $P(C\cap A)$ は
				\begin{align}
					P(A\cap B) &= \frac{1}{6}, \\
					P(B\cap C) &= \frac{1}{6}, \\
					P(C\cap A) &= \frac{1}{6}.
				\end{align}
				したがって
				\begin{align}
					P(A\cap B) &\neq P(A)P(B), \\
					P(B\cap C) &= P(B)P(C), \\
					P(C\cap A) &= P(C)P(A).
				\end{align}
				すなわち,
				\begin{quote}
					$A$ と $B$: \red{互いに独立でない}, \\
					$B$ と $C$: \red{互いに独立である}, \\
					$C$ と $A$: \red{互いに独立である}.
				\end{quote}
				}
		\end{enumerate}
	}
	\item{
		\begin{enumerate}
			\item{
				$A$, $B$ は互いに独立だから
				\begin{align}
					P(A\cap B) = P(A)P(B) = \frac{1}{3}\cdot\frac{2}{5} = \red{\frac{2}{15}}.
				\end{align}
			}
			\item{
				(1) より
				\begin{align}
					P(A\cup B) &= P(A) + P(B) - P(A\cap B) \\
						&= \frac{1}{3} + \frac{2}{5} - \frac{2}{15} = \red{\frac{3}{5}}.
				\end{align}
			}
		\end{enumerate}
	}
	\item{
		1 回だけ赤玉である取り出し方は, ``赤玉, 白玉, 白玉'' の並べ方を考えると $\dfrac{\permu{3}{3}}{\permu{2}{2}} = 3$ 通りである.
		\begin{enumerate}
			\item{
				復元抽出で取り出す場合, それぞれが起こる確率は
				\begin{align}
					\frac{3}{9}\cdot\left(\frac{6}{9}\right)^{2} = \frac{4}{27}.
				\end{align}
				よって, 求める確率は
				\begin{align}
					3\cdot\frac{4}{27} = \red{\frac{4}{9}}.
				\end{align}
			}
			\item{
				非復元抽出で取り出す場合, それぞれが起こる確率は
				\begin{align}
					\frac{3}{9}\cdot\frac{6}{8}\cdot\frac{5}{7} = \frac{5}{28}.
				\end{align}
				よって, 求める確率は
				\begin{align}
					3\cdot\frac{5}{28} = \red{\frac{15}{28}}.
				\end{align}
			}
		\end{enumerate}
	}
	\item{
		\begin{enumerate}
			\item{
				1 の目がちょうど 3 回出る出方は, ``1 の目, 1 の目, 1 の目, 1 の目以外'' の並べ方を考えると $\dfrac{\permu{4}{4}}{\permu{3}{3}} = 4$ 通りである.
				それぞれが起こる確率は
				\begin{align}
					\left(\frac{1}{6}\right)^{3}\cdot\frac{5}{6} = \frac{5}{6^{4}}.
				\end{align}
				よって, 求める確率は
				\begin{align}
					4\cdot\frac{5}{6^{4}} = \red{\frac{5}{324}}.
				\end{align}
			}
			\item{
				表がちょうど 2 回出る出方は, ``表, 表, 裏, 裏, 裏, 裏, 裏'' の並べ方を考えると $\dfrac{\permu{7}{7}}{\permu{2}{2}\cdot\permu{5}{5}} = 21$ 通りである.
				それぞれが起こる確率は
				\begin{align}
					\left(\frac{1}{2}\right)^{2}\cdot\left(\frac{1}{2}\right)^{5} = \frac{1}{128}.
				\end{align}
				よって, 求める確率は
				\begin{align}
					21\cdot\frac{1}{128} = \red{\frac{21}{128}}.
				\end{align}
			}
			\item{
				白玉がちょうど 2 回出る出方は, ``白玉, 白玉, 黒玉'' の並べ方を考えると $\dfrac{\permu{3}{3}}{\permu{2}{2}} = 3$ 通りである.
				それぞれが起こる確率は
				\begin{align}
					\left(\frac{2}{5}\right)^{2}\cdot\frac{3}{5} = \frac{12}{125}.
				\end{align}
				よって, 求める確率は
				\begin{align}
					3\cdot\frac{12}{125} = \red{\frac{36}{125}}.
				\end{align}
			}
		\end{enumerate}
	}
	\item{
		\begin{enumerate}
			\item{
				1 の目が 1 回, 2 の目が 2 回出る出方は, ``1 の目, 2 の目, 2 の目'' の並べ方を考えると $\dfrac{\permu{3}{3}}{\permu{2}{2}} = 3$ 通りである.
				それぞれが起こる確率は
				\begin{align}
					\frac{1}{6}\cdot\left(\frac{1}{6}\right)^{2} = \frac{1}{6^{3}}.
				\end{align}
				よって, 求める確率は
				\begin{align}
					3\cdot\frac{1}{6^{3}} = \red{\frac{1}{72}}.
				\end{align}
			}
			\item{
				3 回とも奇数の目が出る確率は
				\begin{align}
					\left(\frac{3}{6}\right)^{3} = \red{\frac{1}{8}}.
				\end{align}
			}
			\item{
				3 回とも奇数の目が出ない確率は
				\begin{align}
					\left(\frac{3}{6}\right)^{3} = \frac{1}{8}
				\end{align}
				だから, 求める確率は
				\begin{align}
					1 - \frac{1}{8} = \red{\frac{7}{8}}.
				\end{align}
			}
			\item{
				奇数の目が出る回数が偶数の目が出る回数より多くなるのは, 奇数が出る回数が 2 回または 3 回のときである.
				奇数が 2 回出る出方は, ``奇数の目, 奇数の目, 偶数の目'' の並べ方を考えると $\dfrac{\permu{3}{3}}{\permu{2}{2}} = 3$ 通りである.
				それぞれが起こる確率は
				\begin{align}
					\left(\frac{3}{6}\right)^{2}\cdot\frac{3}{6} = \frac{1}{8}.
				\end{align}
				よって, 奇数が 2 回出る確率は
				\begin{align}
					3\cdot\frac{1}{8} = \frac{3}{8}
				\end{align}
				であり, (2) より奇数が 3 回出る確率は $\dfrac{1}{8}$ であるから, 求める確率は
				\begin{align}
					\frac{3}{8} + \frac{1}{8} = \red{\frac{1}{2}}.
				\end{align}
			}
			\item{
				1, 2 回目には 6 以外の目が出ればよいから, 求める確率は
				\begin{align}
					\left(\frac{5}{6}\right)^{2}\cdot\frac{1}{6} = \red{\frac{25}{216}}.
				\end{align}
			}
			\item{
				1 回目または 2 回目に 1 度目の 6 の目が出ればよいから, その出方は 2 通りであり, 確率は
				\begin{align}
					2\cdot\frac{1}{6}\cdot\frac{5}{6} = \frac{5}{18}.
				\end{align}
				この上で 2 回目の 6 の目が出ればよいから, 求める確率は
				\begin{align}
					\frac{5}{18}\cdot\frac{1}{6} = \red{\frac{5}{108}}.
				\end{align}
			}
		\end{enumerate}
	}
\end{qenumerate}

\vspace{\baselineskip}
\check
\begin{qenumerate}
	\item{
		\begin{enumerate}
			\item{
				奇数のカードを引く確率は
				\begin{align}
					P(A) = \frac{25}{50} = \red{\frac{1}{2}}.
				\end{align}
			}
			\item{
				奇数のカードのうち, 3 の倍数は 3, 9, 15, 21, 27, 33, 39, 45 だから
				\begin{align}
					P_{A}(B) = \red{\frac{8}{25}}.
				\end{align}
			}
			\item{
				奇数かつ 3 の倍数のうち, 5 の倍数は 15, 45 だから
				\begin{align}
					P_{A\cap B}(C) = \frac{2}{8} = \red{\frac{1}{4}}.
				\end{align}
			}
			\item{
				奇数かつ 3 の倍数を引く確率 $P(A\cap B)$ は
				\begin{align}
					P(A\cap B) = \frac{8}{50} = \frac{4}{25}.
				\end{align}
				すなわち, 確率の乗法定理より
				\begin{align}
					P(A\cap B\cap C) = P(A\cap B)P_{A\cap B}(C) = \frac{4}{25}\cdot\frac{1}{4} = \red{\frac{1}{25}}.
				\end{align}
			}
		\end{enumerate}
	}
	\item{
		\begin{enumerate}
			\item{
				A が当たりくじを 1 本も引かない確率は
				\begin{align}
					\frac{2}{5}\cdot\frac{1}{4} = \frac{1}{10}
				\end{align}
				だから, 求める確率は
				\begin{align}
					1 - \frac{1}{10} = \red{\frac{9}{10}}.
				\end{align}
			}
			\item{
				A が当たりくじを 2 本引く確率は
				\begin{align}
					\frac{3}{5}\cdot\frac{2}{4} = \frac{3}{10}.
				\end{align}
				この時点のくじの状態は ``3 本のくじの中に当たりくじが 1 本ある'' 状態である.
				この状態から B が 1 本目に当たりくじを引く確率は $\dfrac{1}{3}$ であり, 2 本目に当たりくじを引く確率は
				\begin{align}
					\frac{2}{3}\cdot\frac{1}{2} = \frac{1}{3}
				\end{align}
				である.
				したがって, 求める確率は
				\begin{align}
					\frac{3}{10}\cdot\left(\frac{1}{3} + \frac{1}{3}\right) = \red{\frac{1}{5}}.
				\end{align}
			}
			\item{
				A が当たりくじを 1 本だけ引く引き方は, ``当たりくじ, はずれくじ'' の並べ方を考えると $\permu{2}{2} = 2$ 通りである.
				``当たりくじ, はずれくじ'' の順で引く確率は
				\begin{align}
					\frac{3}{5}\cdot\frac{2}{4} = \frac{3}{10}
				\end{align}
				であり, ``はずれくじ, 当たりくじ'' の順で引く確率は
				\begin{align}
					\frac{2}{5}\cdot\frac{3}{4} = \frac{3}{10}
				\end{align}
				であるから, A が当たりくじを 1 本だけ引く確率は
				\begin{align}
					\frac{3}{10} + \frac{3}{10} = \frac{3}{5}.
				\end{align}
				この時点のくじの状態は ``3 本のくじの中に当たりくじが 2 本ある'' 状態であり, B は必ず 当たりくじを引くことができるから, 求める確率は $\red{\dfrac{3}{5}}$.

				\another
				後半の議論について, A が当たりくじを 1 本だけ引いた状態で B が当たりくじを 1 本も引かない確率は
				\begin{align}
					\frac{1}{3}\cdot\frac{0}{2} = 0
				\end{align}
				となり, B が少なくとも 1 本当たりくじを引く確率は
				\begin{align}
					1 - 0 = 1.
				\end{align}
				したがって, 求める確率は
				\begin{align}
					\frac{3}{5}\cdot 1 = \red{\frac{3}{5}}.
				\end{align}
			}
			\item{
				(1) より, A が当たりくじを 1 本も引かない確率は $\dfrac{1}{10}$ であり, この場合は B は必ず当たりくじを引くことができる.
				また, (2), (3) より, A が当たりくじを引き, B も当たりくじを引く確率は明らかだから, 求める確率は
				\begin{align}
					\frac{1}{10} + \frac{1}{5} + \frac{3}{5} = \red{\frac{9}{10}}.
				\end{align}
			}
		\end{enumerate}
	}
	\item{
		$A$ は, 1 回目の目が 1 であり, 2, 3 回目の目は任意だから $A$ の起こる確率は
		\begin{align}
			P(A) = \frac{1}{6}\cdot\frac{6}{6}\cdot\frac{6}{6} = \frac{1}{6}.
		\end{align}
		$B$ は, 2 回目の目が 1 であり, 1, 3 回目の目は任意だから $B$ の起こる確率は
		\begin{align}
			P(B) = \frac{6}{6}\cdot\frac{1}{6}\cdot\frac{6}{6} = \frac{1}{6}.
		\end{align}
		$C$ は, 3 回続けて 1 の目がでるから $C$ の起こる確率は
		\begin{align}
			P(C) = \frac{1}{6}\cdot\frac{1}{6}\cdot\frac{1}{6} = \frac{1}{6^{3}}.
		\end{align}
		\begin{enumerate}
			\item{
				$A\cap B$ の確率は, 1 回目と 2 回目の目がともに 1 であり, 3 回目の目は任意だから
				\begin{align}
					P(A\cap B) = \frac{1}{6}\cdot\frac{1}{6}\cdot\frac{6}{6} = \frac{1}{6^{2}}.
				\end{align}
				すなわち
				\begin{abstract}
					P(A\cap B) = P(A)P(B)
				\end{abstract}
				が成り立つから $A$ と $B$ は\red{互いに独立である}.
			}
			\item{
				$B\cap C$ の確率は, $B\subset C$ より
				\begin{align}
					P(B\cap C) = P(C) = \frac{1}{6^{3}}.
				\end{align}
				すなわち
				\begin{align}
					P(B\cap C)\neq P(B)P(C)
				\end{align}
				が成り立つから $B$ と $C$ は\red{互いに独立でない}.
			}
			\item{
				$C\cap A$ の確率は, $A\subset C$ より
				\begin{align}
					P(C\cap A) = P(C) = \frac{1}{6^{3}}.
				\end{align}
				すなわち
				\begin{align}
					P(C\cap A)\neq P(C)P(A)
				\end{align}
				が成り立つから $C$ と $A$ は\red{互いに独立でない}.
			}
		\end{enumerate}
	}
	\item{
		\begin{enumerate}
			\item{
				復元抽出で引く場合, 求める確率は
				\begin{align}
					\frac{3}{8}\cdot\frac{3}{8} = \red{\frac{9}{64}}.
				\end{align}
			}
			\item{
				非復元抽出で引く場合, 求める確率は
				\begin{align}
					\frac{3}{8}\cdot\frac{2}{7} = \red{\frac{3}{28}}.
				\end{align}
			}
		\end{enumerate}
	}
	\item{
		\begin{enumerate}
			\item{
				\begin{enumerate}
					\item{
						ちょうど 3 回表が出る場合: \\
						ちょうど 3 回表が出る出方は ``表, 表, 表, 裏, 裏'' の並べ替えを考えると $\dfrac{\permu{5}{5}}{\permu{3}{3}\cdot\permu{2}{2}} = 10$ 通りである.
						それぞれが起こる確率は
						\begin{align}
							\left(\frac{1}{2}\right)^{3}\cdot\left(\frac{1}{2}\right)^{2} = \frac{1}{2^{5}}
						\end{align}
						だから, ちょうど 3 回表が出る確率は
						\begin{align}
							10\cdot\frac{1}{2^{5}} = \frac{10}{2^{5}}.
						\end{align}
					}
					\item{
						ちょうど 4 回表が出る場合: \\
						ちょうど 4 回表が出る出方は ``表, 表, 表, 表, 裏'' の並べ替えを考えると $\dfrac{\permu{5}{5}}{\permu{4}{4}} = 5$ 通りである.
						それぞれが起こる確率は
						\begin{align}
							\left(\frac{1}{2}\right)^{4}\cdot\frac{1}{2} = \frac{1}{2^{5}}
						\end{align}
						だから, ちょうど 4 回表が出る確率は
						\begin{align}
							5\cdot\frac{1}{2^{5}} = \frac{5}{2^{5}}.
						\end{align}
					}
					\item{
						5 回とも表が出る場合: \\
						5 回とも表が出る確率は
						\begin{align}
							\left(\frac{1}{2}\right)^{2} = \frac{1}{2^{5}}.
						\end{align}
					}
				\end{enumerate}
				以上より, 求める確率は
				\begin{align}
					\frac{10}{2^{5}} + \frac{5}{2^{5}} + \frac{1}{2^{5}} = \red{\frac{1}{2}}.
				\end{align}

				\another
				3 回以上表が出る事象を $A$ と表す.
				$A$ の余事象 $\overline{A}$ は ``表が出る回数が 2 回以下'' であり, これは ``裏が 3 回以上出る'' と同値である.
				硬貨を投げて表が出る確率と裏が出る確率は等しいから, $A$ の確率と $\overline{A}$ の確率は等しい.
				以上より
				\begin{align}
					\begin{cases}
						P(A) + P\left(\overline{A}\right) = 1 \\
						P(A) = P\left(\overline{A}\right)
					\end{cases}
					\quad\Leftrightarrow\quad
					P(A) = P\left(\overline{A}\right) = \frac{1}{2}.
				\end{align}
				すなわち, 求める確率は $\red{\dfrac{1}{2}}$.
			}
			\item{
				1 回も裏が出ない (5 回とも表が出る) 確率は (1)-(iii) より $\dfrac{1}{2^{5}}$ だから, 求める確率は
				\begin{align}
					1 - \frac{1}{2^{5}} = \red{\frac{31}{32}}.
				\end{align}
			}
		\end{enumerate}
	}
	\item{
		A, B の 2 人が 1 回じゃんけんをしたとき, A が勝つ確率を考える.
		A は $\dfrac{1}{3}$ の確率で ``グー'' を出し, この場合は B が ``チョキ'' を出したときに勝つことができるから, A が ``グー'' を出して B に勝つ確率は
		\begin{align}
			\frac{1}{3}\cdot\frac{1}{3} = \frac{1}{9}.
		\end{align}
		A が ``チョキ'', ``パー'' を出した場合の確率も同様だから, A が勝つ確率は
		\begin{align}
			\frac{1}{9} + \frac{1}{9} + \frac{1}{9} = \frac{1}{3}.
		\end{align}
		また, B が勝つ確率についても同様に $\dfrac{1}{3}$ である.
		\begin{enumerate}
			\item{
				A が 3 回勝つことによって勝負がつく確率は
				\begin{align}
					\left(\frac{1}{3}\right)^{3} = \frac{1}{27}
				\end{align}
				であり, B についても同様だから, 求める確率は
				\begin{align}
					\frac{1}{27} + \frac{1}{27} = \red{\frac{2}{27}}.
				\end{align}
			}
			\item{
				\begin{enumerate}
					\item{
						4 回目のじゃんけんで勝負がつく確率: \\
						4 回目のじゃんけんで A が勝者になる場合の試行の進み方は, 1 回目から 3 回目のじゃんけんについて ``A の勝ち, A の勝ち, A の勝ち以外\footnote{``A の負け'' または, ``あいこ''.}'' の並べ替えを考えると $\dfrac{\permu{3}{3}}{\permu{2}{2}} = 3$ 通りある.
						それぞれが起こる確率は, 4 回目のじゃんけんで A が勝つ確率も考慮すると
						\begin{align}
							\underbrace{\left(\frac{1}{3}\right)^{2}\cdot\frac{2}{3}}_{\mathclap{\text{1--3 回目のじゃんけん}}}\cdot\overbrace{\frac{1}{3}}^{\mathclap{\text{4 回目のじゃんけん}}} = \frac{2}{81}
						\end{align}
						となるから, 4 回目で A が勝者になる確率は
						\begin{align}
							3\cdot\frac{2}{81} = \frac{2}{27}.
						\end{align}
						一方, 4 回目のじゃんけんで B が勝者になる確率も同様だから, 4 回目のじゃんけんで勝負がつく確率は
						\begin{align}
							\frac{2}{27} + \frac{2}{27} = \frac{4}{27}.
						\end{align}
					}
					\item{
						5 回目のじゃんけんで勝負がつく確率: \\
						5 回目のじゃんけんで A が勝者になる場合の試行の進み方は, 1 回目から 4 回目のじゃんけんについて ``A の勝ち, A の勝ち, A の勝ち以外, A の勝ち以外'' の並べ替えを考えると $\dfrac{\permu{4}{4}}{\permu{2}{2}\cdot\permu{2}{2}} = 6$ 通りある.
						それぞれが起こる確率は, 5 回目のじゃんけんで A が勝つ確率も考慮すると
						\begin{align}
							\underbrace{\left(\frac{1}{3}\right)^{2}\cdot\left(\frac{2}{3}\right)^{2}}_{\mathclap{\text{1--4 回目のじゃんけん}}}\cdot\overbrace{\frac{1}{3}}^{\mathclap{\text{5 回目のじゃんけん}}} = \frac{4}{243}
						\end{align}
						となるから, 5 回目で A が勝者になる確率は
						\begin{align}
							6\cdot\frac{4}{243} = \frac{8}{81}.
						\end{align}
						一方, 5 回目のじゃんけんで B が勝者になる確率も同様だから, 5 回目のじゃんけんで勝負がつく確率は
						\begin{align}
							\frac{8}{81} + \frac{8}{81} = \frac{16}{81}.
						\end{align}
					}
				\end{enumerate}
				求める確率は, 3 回目または 4 回目または 5 回目で勝負がつく確率だから, (1), (i), (ii) より
				\begin{align}
					\frac{2}{27} + \frac{4}{27} + \frac{16}{81} = \red{\frac{34}{81}}.
				\end{align}
			}
		\end{enumerate}
	}
\end{qenumerate}

\vspace{\baselineskip}
\stepup
\begin{qenumerate}
	\item{
		A, B, C のいずれかの袋を任意に選ぶ確率は $\dfrac{1}{3}$ であり, 選んだ袋からそれぞれ白玉を取り出す確率を考える.
		A を選んだ場合は
		\begin{align}
			\frac{1}{3}\cdot\frac{2}{5} = \frac{2}{15}, 
		\end{align}
		B を選んだ場合は
		\begin{align}
			\frac{1}{3}\cdot\frac{1}{5} = \frac{1}{15}, 
		\end{align}
		C を選んだ場合は
		\begin{align}
			\frac{1}{3}\cdot\frac{3}{5} = \frac{3}{15}.
		\end{align}
		これらは互いに排反だから, 求める確率は
		\begin{align}
			\frac{2}{15} + \frac{1}{15} + \frac{3}{15} = \red{\frac{2}{5}}.
		\end{align}
	}
	\item{
		題意より, A の箱を選ぶ確率は $\dfrac{1}{6}$, B の箱を選ぶ確率は $\dfrac{2}{6}$, C の箱を選ぶ確率は $\dfrac{3}{6}$ である.
		選んだ箱から当たりくじを引く確率は, A を選んだ場合は
		\begin{align}
			\frac{1}{6}\cdot\frac{4}{10} = \frac{4}{60}, 
		\end{align}
		B を選んだ場合は
		\begin{align}
			\frac{2}{6}\cdot\frac{3}{10} = \frac{6}{60}, 
		\end{align}
		C を選んだ場合は
		\begin{align}
			\frac{3}{6}\cdot\frac{2}{10} = \frac{6}{60}.
		\end{align}
		これらは互いに排反だから, 求める確率は
		\begin{align}
			\frac{4}{60} + \frac{6}{60} + \frac{6}{60} = \red{\frac{4}{15}}.
		\end{align}
	}
	\item{
		\begin{enumerate}
		\item{
			A, B, C の 3 製品とも不良品である確率は
			\begin{align}
				\frac{1}{100}\cdot\frac{1.5}{100}\cdot\frac{0.5}{100} = \red{\frac{3}{4000000}}.
			\end{align}
		}
		\item{
			A, B のみ不良品であり, C は不良品でない確率は
			\begin{align}
				\frac{1}{100}\cdot\frac{1.5}{100}\cdot\left(1 - \frac{0.5}{100}\right) = \frac{149.25}{1000000}, 
			\end{align}
			B, C のみ不良品であり, A は不良品でない確率は
			\begin{align}
				\frac{1.5}{100}\cdot\frac{0.5}{100}\cdot\left(1 - \frac{1}{100}\right) = \frac{74.25}{1000000}, 
			\end{align}
			C, A のみ不良品であり, B は不良品でない確率は
			\begin{align}
				\frac{0.5}{100}\cdot\frac{1}{100}\cdot\left(1 - \frac{1.5}{100}\right) = \frac{49.25}{1000000}.
			\end{align}
			以上より, 求める確率は
			\begin{align}
				\frac{149.25}{1000000} + \frac{74.25}{1000000} + \frac{49.25}{1000000} = \red{\frac{1091}{4000000}}.
			\end{align}
		}
		\item{
			3 製品とも不良品でない確率は
			\begin{align}
				\left(1 - \frac{1}{100}\right)\left(1 - \frac{1.5}{100}\right)\left(1 - \frac{0.5}{100}\right) = \frac{3881097}{4000000}.
			\end{align}
			したがって, 求める確率は
			\begin{align}
				1 - \frac{3881097}{4000000} = \red{\frac{118903}{4000000}}.
			\end{align}
		}
		\end{enumerate}
	}
	\item{
		選手が 1 回目に命中させる事象を $A$, 2 回目に命中させる事象を $B$ とするとき, 題意より $P_{A}(B) = 0.7$, $P_{\overline{A}}(B) = 0.5$, $P(B) = 0.6$ である.
		ここで, $P(B)$ について
		\begin{align}
			P(B) = P(A\cap B) + P\left(\overline{A}\cap B\right)
		\end{align}
		が成り立ち, 確率の乗法定理より
		\begin{align}
			P(B) &= P(A)P_{A}(B) + P\left(\overline{A}\right)P_{\overline{A}}(B) \\
				&= P(A)P_{A}(B) + (1 - P(A))P_{\overline{A}}(B) \\
				&= P(A)\left(P_{A}(B) - P_{\overline{A}}(B)\right) + P_{\overline{A}}(B)
		\end{align}
		である.
		すなわち
		\begin{align}
			P(A) &= \frac{P(B) - P_{\overline{A}}(B)}{P_{A}(B) - P_{\overline{A}}(B)} = \frac{0.6 - 0.5}{0.7 - 0.5} = \red{0.5}.
		\end{align}
	}
	\item{
		A が当たる事象を $A$, B が当たる事象を $B$ とすると, 題意より $P(A) = \dfrac{b}{a}$ である.
		A が当たりくじを引いた場合, B は ``$a - 1$ 本の中に当たりが $b - 1$ 本含まれているくじ'' を引くことになるから, B が当たりくじを引く確率は $P_{A}(B) = \dfrac{b - 1}{a - 1}$ である.
		A が当たりくじを引かなかった場合, B は ``$a - 1$ 本の中に当たりが $b$ 本含まれているくじ'' をひくことになるから, B が当たりくじを引く確率は $P_{\overline{A}}(B) = \dfrac{b}{a - 1}$ である.
		ここで, $P(B)$ について
		\begin{align}
			P(B) = P(A\cap B) + P\left(\overline{A}\cap B\right)
		\end{align}
		が成り立つから, 確率の乗法定理より
		\begin{align}
			P(B) &= P(A)P_{A}(B) + P\left(\overline{A}\right)P_{\overline{A}}(B) \\
				&= P(A)P_{A}(B) + (1 - P(A))P_{\overline{A}}(B) \\
				&= P(A)\left(P_{A}(B) - P_{\overline{A}}(B)\right) + P_{\overline{A}}(B) \\
				&= \frac{b}{a}\cdot\left(\frac{b - 1}{a - 1} - \frac{b}{a - 1}\right) + \frac{b}{a - 1} \\
				&= \red{\frac{b}{a}}.
		\end{align}
	}
	\item{
		\begin{proof}
			$A$ が起こる確率は $P(A) = \dfrac{1}{2}$ である.
			硬貨を 2 回投げたときの表と裏の組み合わせは, ``表, 表'', ``表, 裏'', ``裏, 表'', ``裏, 裏'' のいずれかになるから, $B$ が起こる確率は $P(B) = \dfrac{3}{4}$ である.
			また, $A$ が起こるとき, $B$ も必ず起こるから $P(A\cap B) = \dfrac{1}{2}$.
			以上より
			\begin{align}
				P(A\cap B) \neq P(A)P(B)
			\end{align}
			が成り立つから, $A$, $B$ は互いに独立でない.
		\end{proof}
	}
	\item{
		\begin{enumerate}
			\item{
				$A$ が起こる確率は $P(A) = \dfrac{2}{4}$, $B$ が起こる確率は $P(B) = \dfrac{2}{4}$ である.
				また, $A\cap B$ が起こる確率は $P(A\cap B) = \dfrac{1}{4}$ であるから
				\begin{align}
					P(A\cap B) = P(A)P(B)
				\end{align}
				が成り立つ.
				すなわち, $A$ と $B$ は\red{互いに独立である}.
			}
			\item{
				$A$ が起こる確率は $P(A) = \dfrac{3}{4}$, $B$ が起こる確率は $P(B) = \dfrac{3}{4}$ である.
				また, $A\cap B$ が起こる確率は $P(A\cap B) = \dfrac{2}{4}$ であるから
				\begin{align}
					P(A\cap B) \neq P(A)P(B)
				\end{align}
				が成り立つ.
				すなわち, $A$ と $B$ は\red{互いに独立でない}.
			}
		\end{enumerate}
	}
	\item{
		\begin{enumerate}
			\item{
				$A$ が起こる確率は $P(A) = \dfrac{21}{36}$, $B$ が起こる確率は $P(B) = \dfrac{18}{36}$ である.
				また, $A\cap B$ が起こる確率は $P(A\cap B) = \dfrac{10}{36}$ であるから
				\begin{align}
					P(A\cap B) \neq P(A)P(B)
				\end{align}
				が成り立つ.
				すなわち, $A$ と $B$ は\red{互いに独立でない}.
			}
			\item{
				$A$ が起こる確率は $P(A) = \dfrac{21}{36}$, $B$ が起こる確率は $P(B) = \dfrac{24}{36}$ である.
				また, $A\cap B$ が起こる確率は $P(A\cap B) = \dfrac{14}{36}$ であるから
				\begin{align}
					P(A\cap B) = P(A)P(B)
				\end{align}
				が成り立つ.
				すなわち, $A$ と $B$ は\red{互いに独立である}.
			}
			\item{
				$A$ が起こる確率は $P(A) = \dfrac{21}{36}$, $B$ が起こる確率は $P(B) = \dfrac{m + n}{36}$ である ($m\leq 21$, $n\leq15$).
				また, $A\cap B$ が起こる確率は $P(A\cap B) = \dfrac{m}{36}$ である.
				$A$ と $B$ が互いに独立であるとき
				\begin{align}
					P(A\cap B) = P(A)P(B)
				\end{align}
				が成り立つから
				\begin{align}
					\frac{m}{36} = \frac{21}{36}\cdot\frac{m + n}{36} \quad&\Leftrightarrow\quad 5m = 7n \\
						&\Leftrightarrow\quad m:n = 7:5.
				\end{align}
				$m\leq 21$, $n\leq15$ より
				\begin{align}
					\red{(7, 5), (14, 10), (21, 15)}.
				\end{align}
			}
		\end{enumerate}
	}
	\item{
		最初に取り出した玉が赤玉である事象を $A_{1}$, 白玉である事象を $A_{2}$ とすると, A として赤玉が入った袋が選ばれる確率と白玉が入った袋が選ばれる確率は等しいから
		\begin{align}
			P\left(A_{1}\right) = P\left(A_{2}\right) = \frac{1}{2}.
		\end{align}
		2 回目に取り出した玉が赤玉である事象を $B$ とすると
		\begin{align}
			B = \left(A_{1}\cap B\right)\cup\left(A_{2}\cap B\right).
		\end{align}
		題意より, $A_{1}$ が起こると A の袋の中は ``赤玉が 5 個, 白玉が 5 個'' の状態だから
		\begin{align}
			P_{A_{1}}(B) = \frac{5}{10}, 
		\end{align}
		$A_{2}$ が起こると A の袋の中は ``赤玉が 3 個, 白玉が 7 個'' の状態だから
		\begin{align}
			P_{a_{2}}(B) = \frac{3}{10}.
		\end{align}
		ここで, $A_{1}\cap B$ と $A_{2}\cap B$ は互いに排反だから
		\begin{align}
			P(B) &= P\left(\left(A_{1}\cap B\right)\cup\left(A_{2}\cap B\right)\right) \\
				&= P\left(A_{1}\cap B\right) + P\left(A_{2}\cap B\right)
		\end{align}
		であり, 確率の乗法定理より
		\begin{align}
			P(B) &= P\left(A_{1}\right)P_{A_{1}}(B) + P\left(A_{2}\right)P_{A_{2}}(B) \\
				&=\frac{1}{2}\cdot\frac{5}{10} + \frac{1}{2}\cdot\frac{3}{10} = \red{\frac{2}{5}}.
		\end{align}
	}
	\item{
		\begin{enumerate}
			\item{
				点 P の $x$ 座標が 3 となるのは, 硬貨を 3 回投げたときである.
				硬貨を 3 回投げて点 P の $y$ 座標が 3 となるのは, 表が 2 回, 裏が 1 回出る場合である.
				この出方は $\dfrac{\permu{3}{3}}{\permu{2}{2}} = 3$ 通りであり, それぞれが起こる確率は
				\begin{align}
					\left(\frac{1}{2}\right)^{2}\cdot\frac{1}{2} = \frac{1}{8}
				\end{align}
				であるから, 求める確率は
				\begin{align}
					3\cdot\frac{1}{8} = \red{\frac{3}{8}}.
				\end{align}
			}
			\item{
				点 P が $(3, 3)$ から $(5, 4)$ を通ることを考える.
				$x$, $y$ 座標の差をそれぞれ $\Delta x$, $\Delta y$ とすると, $\Delta x = 2$, $\Delta y = 1$ である.
				$\Delta x = 2$ より, 点 P の $x$ 座標が 5 となるのは, $(3, 3)$ からさらに硬貨を 2 回投げたときである.
				硬貨を 2 回投げて $\Delta y = 1$ となるのは, 表が 1 回, 裏が 1 回出る場合である.
				この出方は $\permu{2}{2} = 2$ 通りであり, それぞれが起こる確率は
				\begin{align}
					\frac{1}{2}\cdot\frac{1}{2} = \frac{1}{4}
				\end{align}
				であるから, 点 P が $(3, 3)$ から $(5, 4)$ を通る確率は
				\begin{align}
					2\cdot\frac{1}{4} = \frac{1}{2}.
				\end{align}
				(1) より, 点 P が $(3, 3)$ を通る確率は $\dfrac{3}{8}$ だから, 求める確率は
				\begin{align}
					\frac{3}{8}\cdot\frac{1}{2} = \red{\frac{3}{16}}.
				\end{align}
			}
		\end{enumerate}
	}
\end{qenumerate}
