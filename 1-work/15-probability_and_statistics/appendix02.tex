%%%%%%%%%%%%%%%%%%%%%%%%%%%%%%%%%%%%%%%%%%%%%%%%%%%%%%%%%%%%%%%%%%%%%%%%%%%%%%%%%%%%%%%%%%%%%%%%%%%%
%	appendix02
%===================================================================================================
%	AUTHOR	H. NAKAJIMA
%	DATE	2025/01
%===================================================================================================
%	EDIT HISTORY
%	DATE		NAME			OVERVIEW
%---------------------------------------------------------------------------------------------------
%	2025/01		H. NAKAJIMA		Create new
%%%%%%%%%%%%%%%%%%%%%%%%%%%%%%%%%%%%%%%%%%%%%%%%%%%%%%%%%%%%%%%%%%%%%%%%%%%%%%%%%%%%%%%%%%%%%%%%%%%%

\section{$a_{n + 1} = pa_{n} + q$ 型漸化式の一般項}
\label{a_{n + 1} = pa_{n} + q 型漸化式の一般項}

漸化式
\begin{align}
	a_{n + 1} = pa_{n} + q \label{B.特性解型}
\end{align}
で表される数列 $\{a_{n}\}$ の一般項を求めることを考える.
ただし, 初項は $a_{1}$ とする.

\vspace{\baselineskip}
漸化式を
\begin{align}
	a_{n + 1} - \alpha = p(a_{n} - \alpha) \label{B.等比型}
\end{align}
と変形できるとき, これは $b_{n} = a_{n} - \alpha$ とおけば
\begin{align}
	b_{n + 1} = pb_{n} \label{B.等比型2}
\end{align}
となり, 容易に一般項を求めることができる.
$\alpha$ を定めるため, 式~\eqref{B.特性解型} から 式~\eqref{B.等比型} を辺々引けば
\begin{align}
	\begin{array}{cccccccccc}
		        & a_{n + 1} &       &        & {}={} & pa_{n} &       &         & {}+{} & q\\
		-\Bigm) & a_{n + 1} & {}-{} & \alpha & {}={} & pa_{n} & {}-{} & p\alpha &       &  \\ \hline
		        &           &       & \alpha & {}={} &        &       & p\alpha & {}+{} & q
	\end{array}
\end{align}
を得る.
この方程式を {\textbf{特性方程式}} という.
特性方程式を解くと
\begin{align}
	\alpha = \frac{q}{1-p}
\end{align}
となり, 式~\eqref{B.等比型} に代入すると
\begin{align}
	a_{n + 1} - \frac{q}{1 - p} = p\left(a_{n} - \frac{q}{1 - p}\right)
\end{align}
となる.
$b_{n} = a_{n} - \dfrac{q}{1 - p}$ とおくと 式~\eqref{B.等比型2} より一般解は
\begin{align}
	b_{n} = b_{1}p^{n - 1}
\end{align}
となるから, 求める一般解は
\begin{align}
	a_{n} + \frac{q}{1 - p} = \left(a_{1} - \frac{q}{1 - p}\right)p^{n - 1}
	\quad\Leftrightarrow\quad
	a_{n} = \left(a_{1} - \frac{q}{1 - p}\right)p^{n - 1} - \frac{q}{1 - p}
\end{align}
となる.